\documentclass[fleqn, 12pt]{article}
\usepackage[utf8]{inputenc} % default from sharelatex
\usepackage[a4paper, left=30mm, right=30mm, top=30mm, bottom=30mm]{geometry}
\usepackage{indentfirst} % to indent fist paragraph
\usepackage[brazilian]{babel} % BR
\usepackage{amsmath}

\title{Teoria dos números e corpos finitos}

\author{Lucas João Martins}
\date{}

\begin{document}

\maketitle

\section*{}
\subsection*{1. (4.6) For each of the following equations, find an integer $x$
that satisfies the equation.}
  \subsubsection*{a. $5x \equiv 4 \ \ (mod \ \ 3)$}
    \begin{align*}
      & x = 2 \\
      & 5 \times 2 = 10 \\
      & 10 - 4 = 6 = 3 \times 2
    \end{align*}
  \subsubsection*{b. $7x \equiv 6 \ \ (mod \ \ 5)$}
    \begin{align*}
      & x = 3 \\
      & 7 \times 3 = 21 \\
      & 21 - 6 = 15 = 5 \times 3
    \end{align*}
  \subsubsection*{c. $9x \equiv 8 \ \ (mod \ \ 7)$}
    \begin{align*}
      & x = 4 \\
      & 9 \times 4 = 36 \\
      & 36 - 8 = 28 = 7 \times 4
    \end{align*}

\subsection*{2. (4.7) In this text, we assume that the modulus is a positive
integer. But the definition of the expression $a \ \ mod \ \ n$ also makes
 perfect sense if $n$ is negative. Determine the following:}

  Usando $a \ \ mod \ \ n = a - \lfloor a / n \rfloor \times n$.

  \subsubsection*{a. $5 \ \ mod \ \ 3$}
    \begin{align*}
      & 2
    \end{align*}
  \subsubsection*{b. $5 \ \ mod \ \ -3$}
    \begin{align*}
      & 5 - \lfloor 5 / -3 \rfloor \times -3 \\
      & 5 - (-2 \times -3) \\
      & 5 - 6 \\
      & -1 \\
    \end{align*}
  \subsubsection*{c.  $-5 \ \ mod \ \ 3$}
    \begin{align*}
      & -5 - \lfloor -5 / 3 \rfloor \times 3 \\
      & -5 - (-2 \times 3) \\
      & -5 + 6 \\
      & 1 \\
    \end{align*}
  \subsubsection*{d. $-5 \ \ mod \ \ -3$}
    \begin{align*}
      & -5 - \lfloor -5 / -3 \rfloor \times -3 \\
      & -5 - (1 \times -3) \\
      & -5 + 3 \\
      & -2 \\
    \end{align*}

\subsection*{3. (4.8) A modulus of 0 does not fit the definition but is defined
by convention as follows: $a \ \ mod \ \ 0 = a$. With this definition in mind,
what does the following expression mean: $a \equiv b \ \ (mod \ \ 0)$?}

  Significa que $a$ e $b$ são iguais.

\end{document}
