\documentclass[12pt]{article}
\usepackage[utf8]{inputenc} % default from sharelatex
\usepackage[a4paper, left=30mm, right=30mm, top=30mm, bottom=30mm]{geometry}
% dont break my code
\usepackage{indentfirst} % to indent fist paragraph
\usepackage[brazilian]{babel} % BR
\usepackage{listings} % to show code
\usepackage{xcolor} % to colors
\usepackage{hyperref} % to links


\lstset{
  language=Python,
  basicstyle=\ttfamily\small,
  numberstyle=\footnotesize,
  numbers=left,
  backgroundcolor=\color{gray!10},
  frame=single,
  tabsize=2,
  rulecolor=\color{black!30},
  title=\lstname,
  escapeinside={\%*}{*)},
  breaklines=true,
  breakatwhitespace=true,
  framextopmargin=2pt,
  framexbottommargin=2pt,
  extendedchars=false,
  inputencoding=utf8,
  commentstyle=\color{blue},
  keywordstyle=\color{purple},
  showstringspaces=false,
  stringstyle=\color{red}
}

\hypersetup{
  colorlinks=true,
  urlcolor=blue
}

\title{
  Geração de números primos: \\
  \large Miller-Rabin, Fermat e Lucas}

\author{Lucas João Martins}
\date{}

\begin{document}

\maketitle

\section{Códigos das implementações}
\lstinputlisting{mr.py}
\lstinputlisting{fermat.py}
\lstinputlisting{lucas.py}
\lstinputlisting{utils.py}
\lstinputlisting{tests.py}

\section{Sobre a geração}
Com a proposta no enunciado do trabalho: ``Uma forma de se gerar números primos
é primeiro gerar um número aleatório ímpar (grande) e depois testá-lo para saber
se é primo. Caso não seja, gera-se outro número aleatório até que seja primo.''.
Portanto, essa foi a metodologia utilizada para a geração de números primos,
onde a parte de verificação de primalidade foi realizada de três diferentes
formas: Miller-Rabin, Fermat e Lucas.

Para a geração de um número aleatório ímpar (grande) havia duas opções:
\begin{itemize}
  \item trabalhar com o código desenvolvido no trabalho anterior, ou
  \item utilizar o módulo \lstinline{random} do \textit{Python}.
\end{itemize}

Optou-se pela segunda opção, pois acredito que ela possua um melhor desempenho,
já que o módulo já é desenvolvido há vários anos por diversos programadores
diferentes. Além disso, a escolha de Miller-Rabin, Fermat e Lucas foi feita com
base na popularidade desses métodos, pois eles estão entre os que possuem mais
material disponível na internet.

O teste de primalidade de Miller-Rabin foi inicialmente desenvolvido como determinístico por Gary L. Miller, mas depois foi modificado para ser um algoritmo probabilístico por Michael O. Rabin. Ele se baseia na propriedade do pseudoprimo forte.

Já a verificação de primalidade de Fermat é baseada no pequeno teorema de
Fermat. Esse teorema diz que se $n$ é primo e $a$ não é divisível por $p$,
então:
\begin{equation}
  a^{n-1} \equiv 1 \ \ (mod \ \ n)
\end{equation}

Por fim, o teste de Lucas, trabalho do matemático Francês de mesmo nome no
século XIX, também realiza uma verificação probabilística, mas por outro lado
necessita que os fatores primos de $n - 1$ já sejam conhecidos, onde $n$ é o
número que será verificado.

Por se tratarem de verificações probabilísticas e não determinísticas, todos os
testes implementados podem indicar que um número composto é um número primo,
porém a chance disso acontecer pode ser minimizada com o aumento de iterações
nos algoritmos. Com essa característica é importante frisar que os números
gerados são possívelmente primos.

\section{Comparação entre os algoritmos}
Com a verificação do Miller-Rabin foi possível gerar um número primo em todos os
tamanhos solicitados. Já com a verificação de Fermat foi possível gerar um
número primo com quase todos os tamanhos solicitados, só ficou impraticável para
4096 bits, onde ocorreu de se passar 15 minutos e o programa não parar a
execução. Por fim, com a verificação de Lucas só foi possível gerar um número
primo com até 128 bits, porque com valores acima disso (e.g. 168) o tempo ficou
muito longo, mais que 15 minutos, e, também nada do programa parar.

Imagino que o motivo da demora no Fermat esteja associado com a utilização do
\lstinline{math.gcd(a, n)} na linha 42. Já no Lucas, acredito que a lentidão
deve estar associada com a utilização de um módulo externo em
\lstinline{pf.primefac(n)} na linha 38. Por fim, na tabela \ref{table:tests} há
o detalhamento dos testes realizados.

\section{Complexidade dos algoritmos que verificam primalidade}
\begin{itemize}
  \item Miller-Rabin: no pior caso é $O(k * s)$.
  \begin{itemize}
    \item Devido aos laços nas linhas 70 e 78 do código.
  \end{itemize}
  \item Fermat: no pior caso é $O(k * \log n)$.
  \begin{itemize}
    \item Devido ao laço na linha 39 e da chamada de método na linha 42.
  \end{itemize}
  \item Lucas: no pior caso é $O(k * \log n)$.
  \begin{itemize}
    \item Devido aos laços nas linhas 55 e 61 do código. O número de fatores
    primos da linha 61 pode ser aproximado para $\log n$.
  \end{itemize}
\end{itemize}

\section{Código-fonte}
\href{https://github.com/lucasjoao/computer_security/tree/master/t04_prime}
{GitHub}

\section{Referências}
Miller-Rabin:
\begin{itemize}
  \item \href{https://www.cs.cmu.edu/~glmiller/Publications/Papers/Mi76.pdf}
  {Riemann's Hypothesis and Tests for Primality}, Gary L. Miller
  \item \href{https://en.wikipedia.org/wiki/Miller%E2%80%93Rabin_primality_test}
  {Wikipedia}
  \item \href{https://goo.gl/prTCSK}{Algorithm Implementation no Wikibooks}
  \item \href{https://rosettacode.org/wiki/Miller%E2%80%93Rabin_primality_test}
  {Rosetta Code}
\end{itemize}

Fermat:
\begin{itemize}
  \item \href{https://goo.gl/fwCCM3}{Khan Academy}
  \item \href{https://en.wikipedia.org/wiki/Fermat_primality_test}
  {Wikipedia}
\end{itemize}

Lucas:
\begin{itemize}
  \item \href{https://github.com/elliptic-shiho/primefac-fork}
  {Módulo primefac}
  \item \href{https://en.wikipedia.org/wiki/Lucas_primality_test}
  {Wikipedia}
\end{itemize}

\section{Tabelas}
\begin{table}[h]
  \centering
  \caption{1 número primo grande}
  \label{table:tests}
  \begin{tabular}{|c|c|c|}
  \hline
  Gerador & Tamanho do número em bits & Tempo gasto  \\ \hline
  Lucas     & 40 & 0m0,040s \\ \hline
  Fermat     & 40 & 0m0,037s  \\ \hline
  Miller-Rabin     & 40 & 0m0,050s  \\ \hline
  Lucas     & 56 & 0m0,097s \\ \hline
  Fermat     & 56 & 0m0,043s  \\ \hline
  Miller-Rabin     & 56 & 0m0,063s  \\ \hline
  Lucas     & 80 & 0m2,927s \\ \hline
  Fermat     & 80 & 0m0,047s  \\ \hline
  Miller-Rabin     & 80 & 0m0,063s  \\ \hline
  Lucas     & 128 & 12m18,283s \\ \hline
  Fermat     & 128 & 0m0,050s  \\ \hline
  Miller-Rabin     & 128 & 0m0,050s  \\ \hline
  Lucas     & 168 & IMPRATICÁVEL \\ \hline
  Fermat     & 168 & 0m0,050s  \\ \hline
  Miller-Rabin     & 168 & 0m0,077s  \\ \hline
  Fermat     & 224 & 0m0,053s  \\ \hline
  Miller-Rabin     & 224 & 0m0,050s  \\ \hline
  Fermat     & 256 & 0m0,073s  \\ \hline
  Miller-Rabin     & 256 & 0m0,087s  \\ \hline
  Fermat     & 512 & 0m0,493s  \\ \hline
  Miller-Rabin     & 512 & 0m0,390s  \\ \hline
  Fermat     & 1024 & 0m2,693s  \\ \hline
  Miller-Rabin     & 1024 & 0m1,533s \\ \hline
  Fermat     & 2048 & 0m26,563s  \\ \hline
  Miller-Rabin     & 2048 & 0m5,683s \\ \hline
  Fermat     & 4096 & IMPRATICÁVEL  \\ \hline
  Miller-Rabin     & 4096 & 0m18,910s \\ \hline
  \end{tabular}
\end{table}
\end{document}
