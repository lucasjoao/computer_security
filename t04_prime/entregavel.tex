\documentclass[12pt]{article}
\usepackage[utf8]{inputenc} % default from sharelatex
\usepackage[a4paper, left=30mm, right=30mm, top=30mm, bottom=30mm]{geometry}
% dont break my code
\usepackage{indentfirst} % to indent fist paragraph
\usepackage[brazilian]{babel} % BR
\usepackage{listings} % to show code
\usepackage{xcolor} % to colors
\usepackage{hyperref} % to links


\lstset{
  language=Python,
  basicstyle=\ttfamily\small,
  numberstyle=\footnotesize,
  numbers=left,
  backgroundcolor=\color{gray!10},
  frame=single,
  tabsize=2,
  rulecolor=\color{black!30},
  title=\lstname,
  escapeinside={\%*}{*)},
  breaklines=true,
  breakatwhitespace=true,
  framextopmargin=2pt,
  framexbottommargin=2pt,
  extendedchars=false,
  inputencoding=utf8,
  commentstyle=\color{blue},
  keywordstyle=\color{purple},
  showstringspaces=false,
  stringstyle=\color{red}
}

\hypersetup{
  colorlinks=true,
  urlcolor=blue
}

\title{
  Geração de números primos: \\
  \large Miller-Rabin, Fermat e Lucas}

\author{Lucas João Martins}
\date{}

\begin{document}

\maketitle

\section{Códigos das implementações}
\lstinputlisting{mr.py}
\lstinputlisting{fermat.py}
\lstinputlisting{lucas.py}

\section{Explicação dos algoritmos}
% copiar do enunciado como os números serão gerados
% explicação geral
% justificativa
% comparação (verificar como foi feito no anterior)
% comentar sobre não usar trabalho anterior
% colocar código do utils
% colocar código de tests

\section{Comparação entre os algoritmos}
% \begin{table}[h]
%   \centering
%   \caption{5 números pseudo-aleatórios grandes}
%   \begin{tabular}{|c|c|c|}
%   \hline
%   Gerador & Tamanho do número em bits & Tempo gasto  \\ \hline
%   ICG     & 27 & 1m15,687s \\ \hline
%   ICG     & mais que 30 & Muito tempo, impraticável \\ \hline
%   LCG     & 40 & 0m0,023s  \\ \hline
%   LFG     & 40 & 0m0,037s  \\ \hline
%   LCG     & 56 & 0m0,027s  \\ \hline
%   LFG     & 56 & 0m0,043s  \\ \hline
%   LCG     & 80 & 0m0,030s  \\ \hline
%   LFG     & 80 & 0m0,035s  \\ \hline
%   LCG     & 128 & 0m0,023s  \\ \hline
%   LFG     & 128 & 0m0,033s  \\ \hline
%   LCG     & 168 & 0m0,027s  \\ \hline
%   LFG     & 168 & 0m0,040s  \\ \hline
%   LCG     & 512 & 0m0,027s  \\ \hline
%   LFG     & 512 & 0m0,043s  \\ \hline
%   LCG     & 2048 & 0m0,023s  \\ \hline
%   LFG     & 2048 & Overflow \\ \hline
%   LCG     & 4096 & 0m0,023s  \\ \hline
%   \end{tabular}
% \end{table}

% dificuldades
% quanto tempo do seu computador é necessário para se gerar números primos

\section{Complexidade dos algoritmos}
% \begin{itemize}
%   \item Inversive congruential generator: $O(n * q)$
%   \begin{itemize}
%     \item Devido aos laços nas linhas 44 e 74 do código.
%   \end{itemize}
%   \item Lagged Fibonacci generator: $O(n * k)$
%   \begin{itemize}
%     \item Devido aos laços nas linhas 94 e 95 do código.
%   \end{itemize}
%   \item Linear congruential generator: $O(n)$
%     \begin{itemize}
%       \item Devido ao laço na linha 51 do código.
%     \end{itemize}
% \end{itemize}

\section{Referências}

\end{document}
