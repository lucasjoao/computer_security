\documentclass[12pt]{article}
\usepackage[utf8]{inputenc} % default from sharelatex
\usepackage[a4paper, left=30mm, right=30mm, top=30mm, bottom=30mm]{geometry}
\usepackage{indentfirst} % to indent fist paragraph
\usepackage[brazilian]{babel} % BR

\title{Resumo artigo original do Bitcoin:
    \\
    \large{``Bitcoin: A peer-to-peer electronic cash system''} }

\author{Lucas João Martins}
\date{}

\begin{document}

\maketitle

Na época do artigo o comércio na internet estava muito dependente de maneira
exclusiva de instituições financeiras que serviam de ``trusted third party
(TTP)'' para o pagamento eletrônico. Com isso, havia um aumento no custo das
transações. Além disso, isso era uma oportunidade, já que não existia mecanismo
que possibilitasse o pagamento sem a necessidade de um TTP.

Era necessário um sistema de pagamento eletrônico baseado em prova criptográfica
ao inves do TTP. Onde a solução proposta resolvia o problema de duplo pagamento,
que é quando se usa a mesma unidade da moeda para realizar dois pagamentos
diferentes, através de um servidor peer-to-peer distribuído que utiliza do
timestamp para gerar prova computacional da ordem cronológica das transações. O
sistema será seguro enquanto os nodos participantes honestos controlarem
coletivamente o maior poder de CPU.

Uma moeda eletrônica é definida como uma corrente de assinaturas digitais. Cada
proprietário transfere a moeda para o próximo assinando um hash da transação
anterior e a chave pública do próximo proprietário, onde tudo isso é adicionado
no fim da moeda. Qualquer interessado pode verificar as assinaturas.

O problema é que o interessado não consegue verificar se alguém realizou duplo
pagamento. A solução é que as transações precisam ser anunciadas publicamente e
os participantes do sistema precisam concordar com isso. Logo, a corrente com o
maior número de concordância será considerada como a primeira recebida.

A solução do servidor timestamp pega um hash do bloco de itens que receberá o
timestamp e divulga para todos. Cada timestamp possui o timestamp anterior ao
seu hash.

O servidor timestamp é implementado em uma rede peer-to-peer. Onde o
``proof-of-work'' é


\section*{Referência}
NAKAMOTO, Satoshi. Bitcoin: A peer-to-peer electronic cash system. 2008.

\end{document}
